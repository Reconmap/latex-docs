\documentclass{beamer}
\usetheme{CambridgeUS}

\usepackage[utf8]{inputenc}
\usepackage{pgfpages}
\usepackage{outlines}

%\setbeameroption{show notes on second screen}
%\setbeameroption{hide notes}
%\setbeameroption{show only notes}

\title{Generating pentest reports with Reconmap}
\subtitle{CyberScotland Week}
\author{Santiago Lizardo}
\date{February 26, 2021}

\begin{document}

\begin{frame}
    \note[item]{...}
    
    \titlepage
\end{frame}

\begin{frame}
    \frametitle{Outline}

    \tableofcontents
\end{frame}

\section{Basic pentesting concepts}

\begin{frame}
    \frametitle{Penetration testing}

    \note[item]{
    Penetration testing (also called pen testing or ethical hacking) is a systematic process of probing for vulnerabilities in your networks and applications. It is essentially a controlled form of hacking in which the 'attackers' act on your behalf to find and test weaknesses that criminals could exploit.    
    }

	\begin{itemize}
		\item Systematic process
		\item Seeks for vulnerabilities
		\item Controlled and authorised
		\item Prevents malicious attacks
	\end{itemize}	    
\end{frame}

\begin{frame}
    \frametitle{Pentester}

    \note[item]{
    }
\end{frame}

\begin{frame}
    \frametitle{Becoming a pentester}

    \note[item]{
    }
\end{frame}

\begin{frame}
    \frametitle{Pentesting workflow}

    \note[item]{
    }
\end{frame}

\begin{frame}
    \frametitle{Penetration testing standards}

    \note[item]{
    }
    
	\begin{itemize}
		\item OSSTMM
		\item OWASP
		\item NIST
		\item PTES
		\item ISSAF		
	\end{itemize}	    
\end{frame}

\section{Reconmap}

\begin{frame}
    \frametitle{Reconmap intro}

    \note[item]{
    }
\end{frame}

\begin{frame}
    \frametitle{Features}

    \note[item]{
    }
\end{frame}

\begin{frame}
    \frametitle{Workflow}

    \note[item]{
    }
\end{frame}

\begin{frame}
    \frametitle{Recap}

    \note[item]{
    }

    \tableofcontents
\end{frame}

\end{document}
